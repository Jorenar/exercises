\documentclass[12pt]{article}

\usepackage[a4paper, vmargin=10pt, hmargin=50pt]{geometry}
\usepackage{amsmath}
\usepackage{amssymb}
\usepackage{polski}

\usepackage{pgfplots}

\setlength{\parskip}{1em}

\newcommand{\task}[1]{\section*{Zadanie #1}}
\newcommand{\ex}[1]{\textbf{ #1)}}
\newcommand{\qef}{\textbf{\textit{QEF}}}
\newcommand{\qed}{\textbf{\textit{QED}}}

\let\oldref\ref
\renewcommand{\ref}[1]{(\oldref{#1})}

\DeclareMathOperator{\dom}{dom}

\title{Zestaw 7}
\author{Jakub Łukasiewicz}
\date{}

\begin{document}

\maketitle

\task{1}\ex{a}
Sprawdzić, czy funkcja $f(x) = \sin{\pi x}$ spełnia założenia twierdzenia
Rolle'a na przedziale $[-1,1]$. Narysować jej wykres.

1. Funkcje $\sin{\alpha}$ i $\pi x$ są ciągłe. Złożenie funkcji ciągłych jest
funkcją ciągłą. Funkcje te są ciągłe szczególnie w $[-1,1]$.

2. $ f'(x) = \pi \cos{\pi x} \therefore $ funkcja $f$ ma pochodną na $(-1,1)$

3. $f(-1) = f(1) = 0$

Funkcja $f$ spełnia założenia twierdzenia Rolle'a na przedziale $[-1,1]$

Wykres:

\begin{figure}[h!]
    \centering
    \begin{tikzpicture}[>=stealth]
        \begin{axis}[
                xmin=-4,xmax=4,
                ymin=-2,ymax=2,
                axis x line=middle,
                axis y line=middle,
                axis line style=->,
                xlabel={$x$},
                ylabel={$y$},
            ]
            \addplot[blue] expression[domain=-pi:pi,samples=300]{sin(deg(pi*x))};
        \end{axis}
    \end{tikzpicture}
\end{figure}

\task{4}\ex{f}
Znaleźć przedziały monotoniczności podanych funkcji $r(x) = \frac{1}{x \ln{x}}$

$$ \Big( \dom{r} = \mathbb{R_+} \setminus \{1\} \qquad
r'(x) = \frac{\ln{x} + 1}{x^2 \ln^2{x}} \qquad
\forall{x \in \dom{r}} : x^2 \ln^2{x} > 0 \Big) \implies $$

$$
\implies
\left\{ \begin{array}{ll}
    f'(x) < 0   \leftrightarrow x < e   \ \because \ \ln{x} <  -1 &\leftrightarrow x < e   \\
    f'(x) \ge 0 \leftrightarrow x \ge e \ \because \ \ln{x} \ge 1 &\leftrightarrow x \ge e
\end{array} \right.
\implies
$$

$$ \implies f \searrow (0,e),\quad f \nearrow [e,\infty) \qquad\qef $$

\newpage

\task{10}\ex{c}
Napisać wzór Maclaurina dla funkcji $f(x) = e^{\tan{x}}$ z resztą $R_2$

$$ f'(x) = e^{\tan{x}} \sec^2{x} $$
$$ f''(x) = e^{\tan{x}} \sec^2{x} (2 \tan{x} + \sec^2{x}) $$

$$ f(x) = f(0) + \frac{f'(0)}{1!}x + \frac{f''(c)}{2!}x^2 $$
$$ f(x) = e^0 + \frac{e^0 1^2}{1}x + \frac{e^c \sec^2{c}(2 \tan{c} + \sec^2{c})}{2}x^2 $$

$$ f(x) = 1 + x + \frac{e^c \sec^2{c}(2 \tan{c} + \sec^2{c})}{2}x^2 \leftrightarrow c \in (0,x) $$
$$ \qef $$

\end{document}

% vim: ft=tex
