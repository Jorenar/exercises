\documentclass[12pt]{article}

\usepackage[a4paper, vmargin=10pt, hmargin=50pt]{geometry}
\usepackage{amsmath}
\usepackage{amssymb}
\usepackage{polski}

\setlength{\parskip}{1em}

\newcommand{\task}[1]{\section*{Zadanie #1}}
\newcommand{\ex}[1]{\textbf{ #1)}}
\newcommand{\qef}{\textbf{\textit{QEF}}}
\newcommand{\qed}{\textbf{\textit{QED}}}

\DeclareMathOperator{\dom}{dom}
\DeclareMathOperator{\sgn}{sgn}
\DeclareMathOperator{\R}{\mathbb{R}}

\let\oldref\ref
\renewcommand{\ref}[1]{(\oldref{#1})}
\renewcommand{\epsilon}{\varepsilon}

\title{Zestaw 8}
\author{Jakub Łukasiewicz}
\date{}

\begin{document}

\maketitle

\task{1}\ex{c}
Oszacować dokładność $ \sqrt{1+x} \approx 1 + \frac{x}{2} - \frac{x^2}{8} $
przybliżonego dla $ |x| \le \frac{1}{4} $

$$ f(x) = \sqrt{1+x} = (1+x)^{\frac{1}{2}} $$
$$ f'(x) = \frac{1}{2}(1+x)^{-\frac{1}{2}} \cdot 1 = \frac{1}{2\sqrt{1+x}} $$
$$ f''(x) = \frac{1}{2} (-\frac{1}{2})(1+x)^{-\frac{3}{2}} \cdot 1 = -\frac{1}{4}\frac{1}{(1+x)^{\frac{3}{2}}} $$
$$ f'''(x) = -\frac{1}{4} (-\frac{3}{2})(1+x)^{-\frac{5}{2}} = \frac{3}{8} \frac{1}{ (1+x)^{ \frac{5}{2}} }$$

$$ P_2(x) = f(0) + \frac{f'(0)}{1!}x + \frac{f''(0)}{2!}x^2 + R_2(x) $$

$$ P_2(x) = 1 + \frac{x}{2} - \frac{x^2}{8} + R_2(x) $$

$$
R_2(x) = \frac{f'''(c)}{3!}x^3 = \frac{3}{8} \frac{1}{(1+c)^{\frac{5}{2}}} \frac{1}{6} x^3
= \frac{1}{16\sqrt{(1+c)^5}} x^3 \leftrightarrow c \in (0,\frac{1}{4}) \because |x| \le \frac{1}{4}
$$

$$ \textrm{Zatem błąd przybliżenia wynosi } \frac{1}{4^5}
\quad\because R_2(x) < \frac{1}{16\sqrt{(1+0)^5} \cdot 4^3 } $$
$$ \qef $$

\newpage

\task{3}\ex{b}
Korzystając z definicji uzasadnić, że funkcja $ g(x) = x^{20} - 3 $
ma ekstremum lokalne w $ x_0 = 0 $

$\newline$
Funkcja ma ekstremum w punkcie, gdy w tym punkcie ma minimum lub maksimum.

Funkcja ma minimum gdy: \ \ $ \exists{\delta>0} : \forall{x \in S(x_0, \delta)} : f(x) > f(x_0) $

Funkcja ma maksimum gdy:    $ \exists{\delta>0} : \forall{x \in S(x_0, \delta)} : f(x) \le f(x_0) $

$$ g(0) = 0^{20} - 3 = 0 - 3 = -3 $$
$$ \textrm{Niech } \epsilon = \frac{1}{n} \textrm{ dla } n \to \infty
\textrm{, zatem } -\epsilon \in (-\delta, 0) \land \epsilon \in (0,\delta) $$

1$^\circ\textrm{ Minimum:}$
$$ g(\epsilon) = \frac{1}{n^{20}} - 3 > -3 $$
$$ g(-\epsilon) = \frac{1}{(-n)^{20}} - 3 = \frac{1}{n^{20}} - 3 > -3 $$

Zatem isnieje minimum lokalne, ergo funckja ma eksremum lokalne w punkcie $x_0 = 0$
\quad\qed

\task{6}\ex{c}
Określić przedziały wypukłości oraz punkty przegięcia funkcji $ h(x) = \tan{x} $

\textit{W rozważaniach każde} $k \in \mathbb{Z}$

$$ \dom{h} = \{ \alpha \in \R : \alpha \ne \frac{\pi}{2} + k\pi \} $$
$$ h'(x) = \sec^2{x} $$
$$ h''(x) = 2(\tan{x})(\sec^2{x}) $$

$$ h''(x) > 0 \implies 2(\tan{x})(\sec^2{x}) > 0 \implies \tan{x} > 0 $$
$$ \therefore \ h(x) \textrm{ jest wypukła na przedziałach } (0 + k\pi, \frac{\pi}{2} + k\pi) $$

Analogicznie określamy wklęsłość: $ (-\frac{\pi}{2} + k\pi, 0 + k\pi) $

Wyznaczamy punkty przegięcia:
$$ h''(x) = 0 \implies 2(\tan{x})(\sec^2{x}) = 0 \implies \tan{x} = 0 \implies x = 0 + k\pi $$
$$ \sgn(h''(0^- + k\pi)) = \sgn(\tan{0^- + k\pi}) = -1 $$
$$ \sgn(h''(0^+ + k\pi)) = \sgn(\tan{0^+ + k\pi}) =  1 $$
$$ \therefore \ \textrm{ punkty przegięcia to } x_0 = k\pi $$

\textit{Podsumowując:} Funkcja $h(x)$ jest wypukła na przedziałach $ (0 + k\pi, \frac{\pi}{2} + k\pi) $,
    wklęsła\\ na przedziałach $ (-\frac{\pi}{2} + k\pi, 0 + k\pi) $, a punkty
    przegięcia to $k\pi$.

\end{document}

% vim: ft=tex
