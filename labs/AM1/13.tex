% LaTeX
% vim: fen

\documentclass[12pt]{article}

\usepackage[a4paper, vmargin=10pt, hmargin=50pt, head=16pt, includehead]{geometry}
\usepackage{amsmath}     % math
\usepackage{amssymb}     % math symbols
\usepackage{enumitem}    % enumerate/list items
\usepackage{fancyhdr}    % header
\usepackage{mathtools}   % math tools
\usepackage{multicol}    % multiple columns
\usepackage{pgfplots}    % graphs
\usepackage{titlesec}    % alternative section titles
\usepackage{xcolor}      % color text

\newcommand*\diff{\mathop{}\!\mathrm{d}}
\newcommand{\case}[1]{\textrm{#1$^\circ$}}
\newcommand{\degree}{^{\circ}}
\newcommand{\for}{\ \leftrightarrow\ }
\newcommand{\Integral}[4]{\int_{#1}^{#2} \! #3 \, \mathop{}\!\mathrm{d}#4}
\newcommand{\mathcolorbox}[2]{\colorbox{#1}{$\displaystyle #2$}}
\newcommand{\norm}[1]{\left\lVert#1\right\rVert}
\newcommand{\qed}{\textbf{\textit{QED}}}
\newcommand{\qef}{\textbf{\textit{QEF}}}
\newcommand{\task}[1]{\textbf{#1}}

\let\oldref\ref \renewcommand{\ref}[1]{\mathrm{(\oldref{#1})}}

\DeclareMathOperator{\?}{?}
\DeclareMathOperator{\dom}{dom}
\DeclareMathOperator{\im}{im}
\DeclareMathOperator{\lin}{lin}
\DeclareMathOperator{\R}{\mathbb{R}}
\DeclareMathOperator{\sgn}{sgn}

\let\oldepsilon\epsilon \renewcommand{\epsilon}{\varepsilon}

\counterwithin*{equation}{section}  % reset equation counter after each section
\titleformat{\section}{\normalfont\Large\bfseries}{}{0pt}{} % remove section numbering

% \newcommand{\sectionbreak}{\clearpage} % start each section on new page

\setlength{\parskip}{0.7em}
\setlength{\parindent}{0em}

\pgfplotsset{samples=20}

\pagestyle{fancy}
\lhead{Jakub Łukasiewicz}
\chead{Zestaw 13}
\let\Sectionmark\sectionmark
\def\sectionmark#1{\def\Sectionname{#1}\Sectionmark{#1}}
%\rhead{\Sectionname}
\rhead{}

\begin{document}
\section{Zadanie 1d}
\textit{Obliczyć wartości średnie funkcji $p(x) = x\sqrt{1-x^2}$ na przedziale $\left[0, \frac{1}{2} \right]$.}

\subsection*{Rozwiązanie}

\begin{equation*}
    \Integral{}{}{p(x)}{x} = \Integral{}{}{x\sqrt{1-x^2}}{x} =
    \begin{vmatrix}
        \displaystyle t = \sqrt{1-x^2} \\[6pt]
        \displaystyle \diff t = -\frac{x}{\sqrt{1-x^2}}\diff x \\[16pt]
        \displaystyle x \diff x = -\diff t \ \sqrt{1-x^2}
    \end{vmatrix}
    =
    -\Integral{}{}{t^2}{t} =
    -\frac{t^3}{3} + C = \frac{(1-x^2)^{\frac{3}{2}}}{3} + C
\end{equation*}

\begin{equation*}
    \left< p \right>_{[0,\frac{1}{2}]} =
    \frac{1}{\frac{1}{2} - 0} \Integral{0}{\frac{1}{2}}{p(x)}{x} =
    2\left( -\frac{(1- \frac{1}{2^2})^{\frac{3}{2}}}{3} + \frac{(1-0^2)^{\frac{3}{2}}}{3} \right) =
    \frac{2}{3} \left(1 - \left(\frac{3}{4}\right)^{\frac{3}{2}} \right) =
    \frac{2}{3} \left(1 - \frac{3}{8}\sqrt{3} \right)
\end{equation*}

\begin{equation*}
    \left< p \right>_{[0,\frac{1}{2}]} =
    \frac{2}{3} - \frac{\sqrt{3}}{4}
\end{equation*}

\section{Zadanie 2a}
\textit{Uzasadnić równość $ \displaystyle \Integral{-\pi}{\pi}{e^{x^2} \sin{x}}{x} = 0 $
wykorzystując własności całek z funkcji parzystych lub nieparzystych.}

\subsection*{Rozwiązanie}
\begin{equation*}
    f(x) = e^{x^2}\ ;\ \ f(-x) = e^{(-x)^2} = e^{x^2} = f(x)
\end{equation*}

Z powyższego równania wynika, że $e^{x^2}$ jest funkcją parzystą. $\sin{x}$
jest funkcją nieparzystą. Iloczyn funkcji parzystej i nieparzystej jest funkcją nieparzystą,
zatem funkcja $e^{x^2} \sin{x}$ jest nieparzysta.

Przedział $[-\pi, \pi]$ jest symetryczny względem $0$.

Całka oznaczona funkcji nieparzystej na przedziale symetrycznym względem $0$
jest równa $0$, zatem:
\begin{equation*}
    \Integral{-\pi}{\pi}{e^{x^2} \sin{x}}{x} = 0
\end{equation*}
\[\qed\]

\clearpage

\section{Zadanie 3b}
\textit{Dla funkcji $f$ całkowalnej na przedziale $[a,b]$, znaleźć funkcje górnej
granicy całkowania i naszkicować wykresy funkcji $f$ i $F$.}

\begin{equation*}
    F(x) = \Integral{c}{x}{f(t)}{t} \for c \in [a,b]
\end{equation*}
\begin{equation*}
    f(x) =
    \left\{ \begin{aligned}
        x - 2  & \for 0 \le x \le 2 \\
        2x - 4 & \for 2  <  x \le 3
    \end{aligned} \right.
\end{equation*}
\begin{equation*}
    [a,b] = [0,3]\ , \quad c = 0
\end{equation*}

\subsection*{Rozwiązanie}

\begin{equation*}
    F(x) = \Integral{c}{x}{f(t)}{t} = \Integral{0}{x}{f(t)}{t}
\end{equation*}
\[ (2) - 2 = 2(2) - 4 = f(2) = 0 \]

\case{1}: $x \le 2$
\begin{equation*}
    F(x) = \Integral{0}{x}{(t-2)}{t} = \left[ \frac{t^2}{2} - 2t \right]_0^x = \frac{x^2}{2} - 2x
\end{equation*}

\case{2}: $x > 2$
\begin{align*}
    F(x) &= \Integral{0}{2}{(t-2)}{t} + \Integral{2}{x}{2t-4}{t} =
    \left( \frac{2^2}{2} - 2\cdot 2 \right) + \left[ t^2 - 4t \right]_2^x =\\
    &= -2 + \left( ( x^2 - 4x ) - (2^2 - 4\cdot 2) \right) =
    -2 + x^2 - 4x + 4 = x^2 - 4x + 2
\end{align*}

Zatem:

\begin{equation*}
    F(x) =
    \left\{ \begin{aligned}
        & \frac{x^2}{2} - 2x  & \for 0 \le x \le 2 \\
        & x^2 - 4x + 2        & \for 2  <  x \le 3
    \end{aligned} \right.
\end{equation*}

Wykres:

\begin{figure}[h!]
    \centering
    \begin{tikzpicture}[>=stealth]
        \begin{axis}[
                xmin=-0.5,xmax=3.5,
                ymin=-3,ymax=3,
                axis x line=middle,
                axis y line=middle,
                xlabel={$x$},
                ylabel={$y$},
                legend style={at={(0.3,1)}, anchor=north west}]
            ]

            \node[blue,circle,fill,inner sep=1pt] at (axis cs:0,-2) {};
            \addplot[smooth,blue] expression[domain=0:3]{x <= 2 ? x-2 : 2*x - 4};
            \node[blue,circle,fill,inner sep=1pt] at (axis cs:3,2) {};

            \node[red,circle,fill,inner sep=1pt] at (axis cs:0,0) {};
            \addplot[smooth,red]  expression[domain=0:3]{x <= 2 ? (x^2)/2 - 2*x : x^2 - 4*x + 2};
            \node[red,circle,fill,inner sep=1pt] at (axis cs:3,-1) {};

            \legend{$f(x)$, $F(x)$}
        \end{axis}
    \end{tikzpicture}
\end{figure}

\end{document}
