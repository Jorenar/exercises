\documentclass{article}
\usepackage[a4paper, margin={1in, 1in}]{geometry}
\usepackage[utf8]{inputenc}
\usepackage{polski}
\usepackage{mathtools}
\usepackage{amsfonts}
\usepackage{amssymb}
\usepackage{amsmath}
\usepackage{multicol}
\usepackage{paralist}
\usepackage{tabto}
\usepackage{graphicx}
\usepackage{etoolbox}
\usepackage{changepage}
\usepackage{tasks}
\usepackage{pgfplots}
\usepackage{fancyhdr}

\DeclareMathOperator{\arctg}{arctg}
\DeclareMathOperator{\sh}{sh}
\DeclareMathOperator{\ch}{ch}
\DeclareMathOperator{\sgn}{sgn}

\let\arctan\relax
\DeclareMathOperator{\arctan}{arctg}
\let\tan\relax
\DeclareMathOperator{\tan}{tg}

% Dodatkowe deklaracje:

\newcommand{\Because}{\quad\because\quad}
\newcommand{\case}[1]{\textrm{#1$^\circ$}}
\newcommand{\degree}{^{\circ}}
\newcommand{\diff}{\mathop{}\!\mathrm{d}}
\newcommand{\for}{\ \leftrightarrow\ }
\newcommand{\Integral}[4]{\int_{#1}^{#2} \! #3 \, \mathop{}\!\mathrm{d}#4}
\newcommand{\mathcolorbox}[2]{\colorbox{#1}{$\displaystyle #2$}}
\newcommand{\norm}[1]{\left\lVert#1\right\rVert}
\newcommand{\qed}{\textbf{\textit{QED}}}
\newcommand{\qef}{\textbf{\textit{QEF}}}
\newcommand{\task}[1]{\textit{#1}}
\newcommand{\xrowht}[2][0]{\addstackgap[.5\dimexpr#2\relax]{\vphantom{#1}}}
\newcommand{\partderiv}[2]{\frac{\partial #1}{\partial #2}}

\DeclareMathOperator{\?}{?}
\DeclareMathOperator{\dom}{dom}
\DeclareMathOperator{\im}{im}
\DeclareMathOperator{\lin}{lin}
\DeclareMathOperator{\N}{\mathbb{N}}
\DeclareMathOperator{\R}{\mathbb{R}}
\DeclareMathOperator{\Z}{\mathbb{Z}}

\let\oldepsilon\epsilon \renewcommand{\epsilon}{\varepsilon}

% Koniec dodatkowych deklaracji

\pagestyle{fancy}

\lhead{Analiza Matematyczna 2; 2020/2021}
% Tutaj proszę uzupełnić imię i nazwisko
\rhead{Jakub Łukasiewicz}

\begin{document}
\section*{Zadanie 1b}
Korzystając z definicji zbadać różniczkowalność
$\displaystyle f(x,y) = \left\{ \begin{array}{ll}
   (x^2 + y^2) \sin \displaystyle \frac{1}{x^2 + y^2} & \for (x,y) \ne (0,0) \\
   0                                    & \for (x,y) =   (0,0)
\end{array}\right. $ w\,punkcie $(x_0, y_0) = (0,0)$.

\subsection*{Rozwiązanie}

\begin{equation*}
   \begin{aligned}
      \frac{\partial f(0,0)}{\partial x} &= \lim_{\Delta{x} \to 0} \frac{f(0+\Delta{x},0) - f(0,0)}{\Delta{x}} =\\
      &= \lim_{\Delta{x} \to 0} \frac{\Big((\Delta{x})^2 - 0)\Big) \sin{\frac{1}{(\Delta{x})^2 + 0}} }{\Delta{x}} =
      \lim_{\Delta{x} \to 0} \Delta{x} \sin{\frac{1}{(\Delta{x})^2}} = 0 \quad\because\quad \sin{\alpha} \in [-1,1]
   \end{aligned}
\end{equation*}

\vspace{1.5em}
Analogicznie $\displaystyle \partderiv{f(0,0)}{y} = 0 $

\begin{equation*}
   \begin{aligned}
      \lceil\mathrm{warunek}\rfloor = & \lim_{(h,k) \to (0,0)} \frac{
         \displaystyle f(0+h,0+k) - f(0,0) - \partderiv{f(0,0)}{x}h - \partderiv{f(0,0)}{y}k
      }{\sqrt{h^2 + k^2}} = \\[1em]
      &= \lim_{(h,k) \to (0,0)} \frac{ f(h,k) - 0 - 0h - 0k }{\sqrt{h^2 + k^2}} =
      \lim_{(h,k) \to (0,0)} \frac{ f(h,k)}{\sqrt{h^2 + k^2}} = \\[1em]
      &= \lim_{(h,k) \to (0,0)} \frac{ (h^2 + k^2) \sin \displaystyle \frac{1}{h^2 + k^2} }{\sqrt{h^2 + k^2}} =
      \lim_{(h,k) \to (0,0)} \sqrt{h^2 + k^2} \sin \displaystyle \frac{1}{h^2 + k^2} =\\[1em]
      &= 0 \quad\because\quad \sin{\alpha} \in [-1,1] \ \land \ \left( (h,k) \to (0,0) \implies \sqrt{h^2 + k^2} \to 0 \right)
   \end{aligned}
\end{equation*}

\subsection*{Odpowiedź}

Funkcja $f(x,y)$ \textbf{jest różniczkowalna} w punkcie $(x_0, y_0) = (0,0)$

\clearpage

\section*{Zadanie 3b}
Wykorzystując różniczkę funkcji obliczyć przybliżoną wartość wyrażenia $\sqrt[3]{(2.93)^3 + (4.05)^3 + (4.99)^3}$

\subsection*{Rozwiązanie}
\begin{equation*}
   f(x_0 + \Delta{x}, y_0 + \Delta{y}, z_0 + \Delta{z}) \approx
   f(x_0, y_0, z_0)
   + \partderiv{f(x_0, y_0, z_0)}{x} \Delta{x}
   + \partderiv{f(x_0, y_0, z_0)}{y} \Delta{y}
   + \partderiv{f(x_0, y_0, z_0)}{z} \Delta{z}
\end{equation*}

Niech:
\begin{itemize}
   \item[] $F = f(x_0 + \Delta{x}, y_0 + \Delta{y}, z_0 + \Delta{z}) $
   \item[] $ f(x,y,z) = \sqrt[3]{x^3 + y^3 + z^3} $
   \item[] $ (x_0, y_0, z_0) = (3,4,5) $
   \item[] $ \Delta{x} = -0.07,\ \Delta{y} = 0.05,\ \Delta{z} = -0.01$
   \item[] $ S = x^3 + y^3 + z^3 $
\end{itemize}

\begin{equation*}
   f(3,4,5) = \sqrt[3]{3^3 + 4^3 + 5^3} = \sqrt[3]{216} = 6
\end{equation*}

\begin{equation*}
   \partderiv{f}{x} = \Big| a = y^3 + z^3 \Big| =
   (x^2 + a)' \frac{1}{3}(x^3+a)^{-\frac{2}{3}} =
   \frac{x^2}{(x^3+a)^{\frac{2}{3}}} =
   \frac{3x^2}{3(x^3+y^3+z^3)^{\frac{2}{3}}} =
   \frac{x^2}{\sqrt[3]{S^2}}
\end{equation*}

\begin{equation*}
   \partderiv{f}{x}(3,4,5) = \frac{x^2}{\sqrt[3]{S^2}} = \frac{9}{\sqrt[3]{216^2}} = \frac{9}{36}
\end{equation*}

\vspace{2em}
Analogicznie:
\quad
$\displaystyle \partderiv{f}{y}(3,4,5) = \frac{4^2}{36} = \frac{16}{9} $
\quad oraz \quad
$\displaystyle \partderiv{f}{z}(3,4,5) = \frac{5^2}{36} = \frac{25}{36} $
\vspace{1em}

Zatem:
\begin{equation*}
   F \approx 6 + \frac{9}{36}(-0.07) + \frac{16}{36}(0.05) + \frac{25}{36}(-0.01) =
   6 - \frac{-0.63 + 0.75 - 0.25}{36} = 6 - \frac{0.13}{36} = \frac{21587}{3600}
\end{equation*}

\subsection*{Odpowiedź}
\begin{equation*}
   \sqrt[3]{(2.93)^3 + (4.05)^3 + (4.99)^3} \approx \frac{21587}{3600}
\end{equation*}

\clearpage

\section*{Zadanie 4b}
Krawędzie prostopadłościanu mają długości $a=3$ m, $b=4$ m, $c=12$ m. Obliczyć
w przybliżeniu, jak zmieni się długość przekątnej prostopadłościanu $d$,
jeżeli długości każdej krawędzi zwiększymy o 2\,cm.

\subsection*{Rozwiązanie}

Przyrost $\Delta{D}$ wartości funkcji $D = D(a,b,c)$ w punkcie $(a,b,c)$ odpowiadający
przyrostom $\Delta{a}$, $\Delta{b}$, $\Delta{c}$ można przybliżyć różniczką tej
funkcji.

\begin{equation*}
   \Delta{D} \approx \partderiv{D(a,b,c)}{a}\Delta{a} + \partderiv{D(a,b,c)}{b}\Delta{b} + \partderiv{D(a,b,c)}{c}\Delta{c}
\end{equation*}

Niech:
\begin{itemize}
   \item[] $D(a,b,c) = \sqrt{a^2 + b^2 + c^2}$
   \item[] $a = 300 \mathrm{\ cm}$, $b = 400 \mathrm{\ cm}$, $c = 1200 \mathrm{\ cm}$
   \item[] $p = \Delta{a} = \Delta{b} = \Delta{c} = 2 \mathrm{cm}$
\end{itemize}

\begin{equation*}
   \partderiv{D}{a} = \Big| A = b^2 + c^2 \Big| = (a^2 + A)'\frac{1}{2}(a+A)^{-\frac{1}{2}}
   = \frac{a}{\sqrt{a^2 + A}} = \frac{a}{\sqrt{a^2 + b^2 + c^2}}
\end{equation*}

\begin{equation*}
   \partderiv{D}{a}(300,400,1200) = \frac{300}{\sqrt{300^2 + 400^2 + 1200^2}} = \frac{300}{1300} = \frac{3}{13}
\end{equation*}

\vspace{1em}
Analogicznie: \quad $\displaystyle
\partderiv{D}{b}(300,400,1200) = \frac{4}{13}
\quad\land\quad
\partderiv{D}{b}(300,400,1200) = \frac{12}{13}$

\vspace{1em}
Zatem:

\begin{equation*}
   \Delta{D} \approx \frac{3}{13} \cdot 2 + \frac{4}{13} \cdot 2 + \frac{12}{13} \cdot 2
   = 2\left(\frac{3+4+12}{13}\right) = \frac{38}{13}
\end{equation*}

\subsection*{Odpowiedź}

Długość przekątnej $d$ zwiększy się w przybliżeniu o $\dfrac{38}{13}$ cm.

\end{document}
