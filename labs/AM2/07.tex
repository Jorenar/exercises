\documentclass{article}
\usepackage[a4paper, margin={1in, 1in}]{geometry}
\usepackage[utf8]{inputenc}
\usepackage{polski}
\usepackage{mathtools}
\usepackage{amsfonts}
\usepackage{amssymb}
\usepackage{amsmath}
\usepackage{multicol}
\usepackage{paralist}
\usepackage{tabto}
\usepackage{graphicx}
\usepackage{etoolbox}
\usepackage{changepage}
\usepackage{tasks}
\usepackage{pgfplots}
\usepackage{fancyhdr}

\DeclareMathOperator{\arctg}{arctg}
\DeclareMathOperator{\sh}{sh}
\DeclareMathOperator{\ch}{ch}
\DeclareMathOperator{\sgn}{sgn}

\let\arctan\relax
\DeclareMathOperator{\arctan}{arctg}
\let\tan\relax
\DeclareMathOperator{\tan}{tg}

% Dodatkowe deklaracje:

\newcommand{\Because}{\quad\because\quad}
\newcommand{\case}[1]{\textrm{#1$^\circ$}}
\newcommand{\degree}{^{\circ}}
\newcommand{\diff}{\mathop{}\!\mathrm{d}}
\newcommand{\for}{\ \leftrightarrow\ }
\newcommand{\Integral}[4]{\int_{#1}^{#2} \! #3 \, \mathop{}\!\mathrm{d}#4}
\newcommand{\mathcolorbox}[2]{\colorbox{#1}{$\displaystyle #2$}}
\newcommand{\norm}[1]{\left\lVert#1\right\rVert}
\newcommand{\qed}{\textbf{\textit{QED}}}
\newcommand{\qef}{\textbf{\textit{QEF}}}
\newcommand{\task}[1]{\textit{#1}}
\newcommand{\xrowht}[2][0]{\addstackgap[.5\dimexpr#2\relax]{\vphantom{#1}}}
\newcommand{\partderiv}[2]{\frac{\partial #1}{\partial #2}}

\DeclareMathOperator{\?}{?}
\DeclareMathOperator{\dom}{dom}
\DeclareMathOperator{\e}{e}
\DeclareMathOperator{\im}{im}
\DeclareMathOperator{\lin}{lin}
\DeclareMathOperator{\N}{\mathbb{N}}
\DeclareMathOperator{\R}{\mathbb{R}}
\DeclareMathOperator{\Z}{\mathbb{Z}}

\let\oldepsilon\epsilon \renewcommand{\epsilon}{\varepsilon}

% Koniec dodatkowych deklaracji

\pagestyle{fancy}

\lhead{Analiza Matematyczna 2; 2020/2021}
% Tutaj proszę uzupełnić imię i nazwisko
\rhead{Jakub Łukasiewicz}

\begin{document}
\section*{Zadanie 3e}

Napisać wzór Taylora z resztą $R_3$ dla funkcji $f(x,y) = \e^{x+2y}$
w otoczeniu punktu ${(x_0, y_0) = (0,0)}$

\subsection*{Rozwiązanie}

Wzór Taylora w punkcie $(x_0,y_0)$ dla funkcji $f(x,y)$ i $n=3$ ma postać:

\begin{equation*}
   \begin{aligned}
      f(x,y) &= f(x_0, y_0) + \frac{1}{1!}\left[ \partderiv{f(x_0,y_0)}{x} \cdot (x-x_0) +\partderiv{f(x_0,y_0)}{y} \cdot (y-y_0) \right] \\
      &\qquad + \frac{1}{2!}\left[
         \frac{\partial^2 f(x_0,y_0)}{\partial x^2} \cdot (x-x_0)^2
         + \frac{\partial^2 f(x_0,y_0)}{\partial x \partial y} \cdot (x-x_0)(y-y_0)
         + \frac{\partial^2 f(x_0,y_0)}{\partial y^2} \cdot (y-y_0)^2
         \right] \\[0.5em]
      &\qquad + R_3(x,y)
   \end{aligned}
\end{equation*}

Gdzie reszta:

\begin{equation*}
   \begin{aligned}
      R_3(x,y) =
      \frac{1}{3!} & \left[
         \frac{\partial^3 f(a,b)}{\partial x^3} \cdot (x-x_0)^3 \right.
         + \frac{\partial^3 f(a,b)}{\partial^2 x \partial y} \cdot (x-x_0)^2(y-y_0) \\
         &\qquad + \frac{\partial^3 f(a,b)}{\partial x \partial^2 y} \cdot (x-x_0)(y-y_0)^2
         + \left. \frac{\partial^3 f(a,b)}{\partial y^3} \cdot (y-y_0)^3
         \right]
   \end{aligned}
\end{equation*}

Gdzie ${(a,b) = (x_0 + c(x-x_0), y_0 + c(y-y_0)) \for 0 < c < 1}$.

\vspace{2em}

Można wykazać, że
$ \displaystyle
\frac{\partial^{k+m}}{\partial x^k \partial y^m}(\e^{ax + by}) := a^k b^m \e^{ax + by}
$, zatem pochodne:

\begin{equation*}
   f(x,y) = \partderiv{f(x,y)}{x} =
   \frac{\partial^2 f(x,y)}{\partial x^2} =
   \frac{\partial^3 f(x,y)}{\partial x^3} = \e^{x+2y}
\end{equation*}

\begin{equation*}
   \frac{\partial^k f(x,y)}{\partial y^k} = 2^k \e^{x+2y}
\end{equation*}

\begin{equation*}
   \frac{\partial^2 f(x,y)}{\partial x \partial y} =
   \frac{\partial^3 f(x,y)}{\partial x^2 \partial y} = 2 \e^{x+2y}
\end{equation*}

\begin{equation*}
   \frac{\partial^3 f(x,y)}{\partial x \partial y^2} = 4 \e^{x+2y}
\end{equation*}

Podstawiając za współrzędne:
\begin{equation*}
   \e^{x_0 + 2y_0} = \e^{0 + 0\cdot 0} = \e^0 = 1
\end{equation*}

Zatem:
\begin{equation*}
   \begin{aligned}
      f(x,y)
      &= 1 + \frac{1}{1!}\big[1(x-0) + 2(y-0)\big] + \frac{1}{2!}\big[1(x-0)^2 + 2(x-0)(y-0) + 4(y-0)^2\big] + R_3(x,y) = \\
      &= 1 + x + 2y + \frac{1}{2}(x^2 + 2xy + 4y^2) + R_3(x,y)
   \end{aligned}
\end{equation*}

Gdzie
\begin{equation*}
   \begin{aligned}
      R_3(x,y)
      &= \frac{1}{3!}\big[1(x-0)^3 + 2(x-0)^2(y-0) + 4(x-0)(y-0)^2 + 8(y-0)^3\big] = \\
      &= \frac{1}{6}(x^3 + 2x^2 y + 4xy^2 + 8y^3)
   \end{aligned}
\end{equation*}

\subsection*{Odpowiedź}
\begin{equation*}
   f(x,y) = 1 + x + 2y + \frac{1}{2}(x^2 + 2xy + 4y^2)
   + \frac{1}{6}(x^3 + 2x^2 y + 4xy^2 + 8y^3)
\end{equation*}

\section*{Zadanie 4a}

Zbadać, czy funkcja $f(x,y) = 2|x| + 3|y|$ ma ekstrema lokalne

\subsection*{Rozwiązanie}

\begin{equation*}
   f(x,y) = 2|x| + 3|y| = 2\sqrt{x^2} + 3\sqrt{y^2}
\end{equation*}

\begin{equation*}
   \partderiv{f}{x} = \left(2\sqrt{x^2}\right)' + 0 = 2\frac{x}{\sqrt{x^2}} = \frac{2x}{|x|}
\end{equation*}

\begin{equation*}
   \textrm{Analogicznie: } \partderiv{f}{y} = \frac{3y}{|y|}
\end{equation*}

Zatem z postaci pierwszych pochodnych cząstkowych i z warunku koniecznego
istnienia ekstremum, wnioskujemy, że może ono występować jedynie w punkcie $(0,0)$.

\begin{equation*}
   f(0,0) = 2|0| + 3|0| = 0 + 0 = 0
\end{equation*}

Z definicji:
\begin{equation*}
   \exists_{\delta > 0} \forall_{(x,y)} \Bigg[
      \Big( (x,y) \in S\big((0,0), \delta \big) \Big) \implies \Big( f(x,y) \ge f(0,0) = 0 \Big)
      \Bigg]
\end{equation*}

Ponieważ brane są wartości bezwzględne zmiennych $x$ i $y$, powyższa implikacja
jest prawdziwa.

\subsection*{Odpowiedź}

Funkcja $f$ ma ekstremum \big(minimum w punkcie $(0,0)$\big)

\section*{Zadanie 5b}

Znaleźć ekstrema funkcji $f(x,y) = x^3 + y^3 - 3xy$

\subsection*{Rozwiązanie}

\begin{equation*}
   \partderiv{f(x,y)}{x} = 3x^2 + 0 - 3y = 3x^2 - 3y
\end{equation*}
\begin{equation*}
   \partderiv{f(x,y)}{x} = 0 + 3y^2 - 3x = 3y^2 - 3x
\end{equation*}

Wobec warunku koniecznego $\left(\displaystyle \partderiv{f}{x} = 0 \ \land \ \partderiv{f}{y} = 0 \right)$
funkcja ta może mieć ekstrema tylko w punktach, w których:
\begin{equation*}
   \partderiv{f(x,y)}{x} = 3x^2 - 3y = 0
   \quad\land\quad
   \partderiv{f(x,y)}{x} = 3y^2 - 3x = 0
\end{equation*}
czyli w $(0,0)$ lub $(1,1)$.

\begin{multicols}{2}
   \noindent
   \begin{equation*}
      \frac{\partial^2 f(x,y)}{\partial x^2} = 6x
   \end{equation*}

   \begin{equation*}
      \frac{\partial^2 f(x,y)}{\partial x \partial y} = \partderiv{}{y}(3x^2 - 3y) = -3
   \end{equation*}

   \begin{equation*}
      \frac{\partial^2 f(x,y)}{\partial y^2} = 6y
   \end{equation*}

   \begin{equation*}
      \frac{\partial^2 f(x,y)}{\partial x \partial y} = \partderiv{}{x}(3y^2 - 3x) = -3
   \end{equation*}
\end{multicols}

\vspace{1em}
Dla punktu $(0,0)$:
$
\displaystyle
   \begin{vmatrix}
      0  & -3 \\
      -3  & 0 \\
   \end{vmatrix} = -9 < 0 \implies \textrm{brak ekstremum}
$

Dla punktu $(1,1)$:
$
\displaystyle
   \begin{vmatrix}
      6  & -3 \\
      -3  & 6 \\
   \end{vmatrix} = 27 > 0 \quad\land\quad \frac{\partial^2 f(1,1)}{\partial x^2} = 6 > 0 \implies \textrm{minimum}
$

\subsection*{Odpowiedź}

Funkcja ma minimum lokalne w punkcie $(1,1)$

\end{document}
