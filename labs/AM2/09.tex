\documentclass{article}
\usepackage[a4paper, margin={1in, 60pt}]{geometry}
\usepackage[utf8]{inputenc}
\usepackage{polski}
\usepackage{mathtools}
\usepackage{amsfonts}
\usepackage{amssymb}
\usepackage{amsmath}
\usepackage{multicol}
\usepackage{paralist}
\usepackage{tabto}
\usepackage{graphicx}
\usepackage{etoolbox}
\usepackage{changepage}
\usepackage{tasks}
\usepackage{pgfplots}
\usepackage{fancyhdr}

\DeclareMathOperator{\arctg}{arctg}
\DeclareMathOperator{\sh}{sh}
\DeclareMathOperator{\ch}{ch}
\DeclareMathOperator{\sgn}{sgn}

\let\arctan\relax
\DeclareMathOperator{\arctan}{arctg}
\let\tan\relax
\DeclareMathOperator{\tan}{tg}

% Dodatkowe deklaracje:

\newcommand{\Because}{\quad\because\quad}
\newcommand{\case}[1]{\textrm{#1$^\circ$}}
\newcommand{\degree}{^{\circ}}
\newcommand{\diff}{\mathop{}\!\mathrm{d}}
\newcommand{\for}{\ \leftrightarrow\ }
\newcommand{\Integral}[4]{\int_{#1}^{#2} \! #3 \, \mathop{}\!\mathrm{d}#4}
\newcommand{\mathcolorbox}[2]{\colorbox{#1}{$\displaystyle #2$}}
\newcommand{\norm}[1]{\left\lVert#1\right\rVert}
\newcommand{\qed}{\textbf{\textit{QED}}}
\newcommand{\qef}{\textbf{\textit{QEF}}}
\newcommand{\task}[1]{\textit{#1}}
\newcommand{\xrowht}[2][0]{\addstackgap[.5\dimexpr#2\relax]{\vphantom{#1}}}
\newcommand{\partderiv}[2]{\frac{\partial #1}{\partial #2}}

\DeclareMathOperator{\?}{?}
\DeclareMathOperator{\dom}{dom}
\DeclareMathOperator{\im}{im}
\DeclareMathOperator{\lin}{lin}
\DeclareMathOperator{\N}{\mathbb{N}}
\DeclareMathOperator{\R}{\mathbb{R}}
\DeclareMathOperator{\Z}{\mathbb{Z}}

\let\oldepsilon\epsilon \renewcommand{\epsilon}{\varepsilon}

% Koniec dodatkowych deklaracji

\pagestyle{fancy}

\lhead{Analiza Matematyczna 2; 2020/2021}
% Tutaj proszę uzupełnić imię i nazwisko
\rhead{Jakub Łukasiewicz}

\begin{document}
\section*{Zadanie 3c}
Całkę podwójną $\displaystyle \iint_D f(x,y) \diff{x}\diff{y}$ zamienić na całki
iterowane, jeżeli obszar $D$ ograniczony jest krzywą o równaniu: $x^2 - 4x + y^2 + 6y - 51 = 0$

\subsection*{Rozwiązanie}

\begin{figure}[h!]
   \centering
   \begin{tikzpicture}
      \begin{axis}[
            xmin=-7,xmax=9,
            ymin=-13,ymax=8,
            axis x line=middle,
            axis y line=middle,
            axis equal,
            xlabel={$x$},
            ylabel={$y$},
         ]
         \draw (axis cs: 2, -3) circle [radius=80];
      \end{axis}
   \end{tikzpicture}
\end{figure}

\begin{equation*}
   \begin{gathered}
      x^2 - 4x + y^2 + 6y - 51 = 0 \\
      (x^2 - 4x + 4) - 4 + (y^2 + 6y + 9) - 9 = 51 \\
      (x - 2)^2 + (y + 3)^2 = 64 \\
   \end{gathered}
\end{equation*}

\subsubsection*{Wyznaczenie dolnej i górnej funkcji}

\begin{equation*}
   \begin{gathered}
      (y + 3)^2 = 64 - (x - 2)^2 \\
      y = -3 \pm \sqrt{64 - (x - 2)^2}\ , \quad -6 \le x \le 10 \\
      g(x) = -3 - \sqrt{64 - (x - 2)^2}\ , \quad h(x) = -3 + \sqrt{64 - (x - 2)^2}\ , \quad -6 \le x \le 10 \\
   \end{gathered}
\end{equation*}

$D$ to obszar normalny względem osi $Ox$, bo:
\begin{equation*}
   D = \{ (x,y) : -6 \le x \le 10,\ -3 - \sqrt{64 - (x - 2)^2} \le y \le -3 + \sqrt{64 - (x - 2)^2} \}
\end{equation*}

Zatem:
\begin{equation*}
   \iint_D f(x,y) \diff{x}\diff{y} = \Integral{-6}{10}{}{x} \Integral{-3 - \sqrt{64 - (x - 2)^2}}{-3 + \sqrt{64 - (x - 2)^2}}{f(x,y)}{y}
\end{equation*}

\subsubsection*{Wyznaczenie lewej i prawej funkcji}

\begin{equation*}
   \begin{gathered}
      (x - 2)^2 = 64 - (y + 3)^2 \\
      x - 2 = \pm\sqrt{64 - (y + 3)^2} \\
      x = 2 \pm \sqrt{64 - (y + 3)^2}\ , \quad -11 \le y \le 5 \\
      p(y) = 2 - \sqrt{64 - (y + 3)^2}\ , \quad q(y) = 2 + \sqrt{64 - (y + 3)^2}\ , \quad -11 \le y \le 5 \\
   \end{gathered}
\end{equation*}

$D$ to obszar normalny względem osi $Oy$, bo:
\begin{equation*}
   D = \{ (x,y) : -11 \le y \le 5,\ 2 - \sqrt{64 - (y + 3)^2} \le y \le 2 + \sqrt{64 - (y + 3)^2} \}
\end{equation*}

Zatem:
\begin{equation*}
   \iint_D f(x,y) \diff{x}\diff{y} = \Integral{-11}{5}{}{y} \Integral{2 - \sqrt{64 - (y + 3)^2}}{2 + \sqrt{64 - (y + 3)^2}}{f(x,y)}{x}
\end{equation*}

\subsection*{Odpowiedź}
\begin{equation*}
   \iint_D f(x,y) \diff{x}\diff{y} = \Integral{-6}{10}{}{x} \Integral{3 - \sqrt{64 - (x - 2)^2}}{3 + \sqrt{64 - (x - 2)^2}}{f(x,y)}{y}
    = \Integral{-11}{5}{}{y} \Integral{2 - \sqrt{64 - (y + 3)^2}}{2 + \sqrt{64 - (y + 3)^2}}{f(x,y)}{x}
\end{equation*}

\end{document}
