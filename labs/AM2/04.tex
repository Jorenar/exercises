\documentclass{article}
\usepackage[a4paper, margin={1in, 1in}]{geometry}
\usepackage[utf8]{inputenc}
\usepackage{polski}
\usepackage{mathtools}
\usepackage{amsfonts}
\usepackage{amssymb}
\usepackage{amsmath}
\usepackage{multicol}
\usepackage{paralist}
\usepackage{tabto}
\usepackage{graphicx}
\usepackage{etoolbox}
\usepackage{changepage}
\usepackage{tasks}
\usepackage{pgfplots}
\usepackage{fancyhdr}

\DeclareMathOperator{\arctg}{arctg}
\DeclareMathOperator{\sh}{sh}
\DeclareMathOperator{\ch}{ch}
\DeclareMathOperator{\sgn}{sgn}

\let\arctan\relax
\DeclareMathOperator{\arctan}{arctg}
\let\tan\relax
\DeclareMathOperator{\tan}{tg}

% Dodatkowe deklaracje:

\newcommand{\Because}{\quad\because\quad}
\newcommand{\case}[1]{\textrm{#1$^\circ$}}
\newcommand{\degree}{^{\circ}}
\newcommand{\diff}{\mathop{}\!\mathrm{d}}
\newcommand{\for}{\ \leftrightarrow\ }
\newcommand{\Integral}[4]{\int_{#1}^{#2} \! #3 \, \mathop{}\!\mathrm{d}#4}
\newcommand{\mathcolorbox}[2]{\colorbox{#1}{$\displaystyle #2$}}
\newcommand{\norm}[1]{\left\lVert#1\right\rVert}
\newcommand{\qed}{\textbf{\textit{QED}}}
\newcommand{\qef}{\textbf{\textit{QEF}}}
\newcommand{\task}[1]{\textit{#1}}
\newcommand{\xrowht}[2][0]{\addstackgap[.5\dimexpr#2\relax]{\vphantom{#1}}}

\DeclareMathOperator{\?}{?}
\DeclareMathOperator{\dom}{dom}
\DeclareMathOperator{\im}{im}
\DeclareMathOperator{\lin}{lin}
\DeclareMathOperator{\N}{\mathbb{N}}
\DeclareMathOperator{\R}{\mathbb{R}}
\DeclareMathOperator{\Z}{\mathbb{Z}}

\let\oldepsilon\epsilon \renewcommand{\epsilon}{\varepsilon}

% Koniec dodatkowych deklaracji

\pagestyle{fancy}

\lhead{Analiza Matematyczna 2; 2020/2021}
% Tutaj proszę uzupełnić imię i nazwisko
\rhead{Jakub Łukasiewicz}

\begin{document}
\section*{Zadanie 1c}
\task{Zbiór $C=\{(x,y,z) \in R^3 : x^2 + y^2 + z^2 < 9\}$ jest ograniczony,
otwarty, domknięty? Czy jest obszarem?}

\subsection*{Rozwiązanie}

Zbiór ten ma kształt kuli, jest zatem ograniczony. Jest też otwarty, ponieważ
$ \forall_{P \in C} \ \exists_{r > 0} : O(P, r) \ne \varnothing $. Nie jest
domknięty, ponieważ nie zawiera sfery o promieniu $3$ ("brzegu"). Kula jest też
obszarem.

\subsection*{Odpowiedź}

Zbiór jest ograniczony, otwarty, nie jest domknięty i jest obszarem.

\vspace{3em}

\section*{Zadanie 4b}
\task{Czy zbiór punktów w przestrzeni
$\displaystyle (x_n, y_n, z_n) = \left( \frac{n^2}{n^2 +1}, \sqrt[n]{2}, 3 \right)$
jest zbieżny? Jeśli tak, to wskazać\,granice}

\subsection*{Rozwiązanie}

\begin{align*}
   \lim_{n \to \infty} x_n &= \lim_{n \to \infty} \frac{n^2}{n^2 + 1} = 1 \\
   \lim_{n \to \infty} y_n &= \lim_{n \to \infty} \sqrt[n]{2}         = 1 \\
   \lim_{n \to \infty} z_n &= \lim_{n \to \infty} 3                   = 3
\end{align*}

\subsection*{Odpowiedź}

Jest zbieżny do $(1,1,3)$

\vspace{3em}

\end{document}
