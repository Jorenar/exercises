\documentclass{article}
\usepackage[a4paper, margin={1in, 1in}]{geometry}
\usepackage[utf8]{inputenc}
\usepackage{polski}
\usepackage{mathtools}
\usepackage{amsfonts}
\usepackage{amssymb}
\usepackage{amsmath}
\usepackage{multicol}
\usepackage{paralist}
\usepackage{tabto}
\usepackage{graphicx}
\usepackage{etoolbox}
\usepackage{changepage}
\usepackage{tasks}
\usepackage{pgfplots}
\usepackage{fancyhdr}

\DeclareMathOperator{\arctg}{arctg}
\DeclareMathOperator{\sh}{sh}
\DeclareMathOperator{\ch}{ch}
\DeclareMathOperator{\sgn}{sgn}

\let\arctan\relax
\DeclareMathOperator{\arctan}{arctg}
\let\tan\relax
\DeclareMathOperator{\tan}{tg}

% Dodatkowe deklaracje:

\newcommand{\Because}{\quad\because\quad}
\newcommand{\case}[1]{\textrm{#1$^\circ$}}
\newcommand{\degree}{^{\circ}}
\newcommand{\diff}{\mathop{}\!\mathrm{d}}
\newcommand{\for}{\ \leftrightarrow\ }
\newcommand{\Integral}[4]{\int_{#1}^{#2} \! #3 \, \mathop{}\!\mathrm{d}#4}
\newcommand{\mathcolorbox}[2]{\colorbox{#1}{$\displaystyle #2$}}
\newcommand{\norm}[1]{\left\lVert#1\right\rVert}
\newcommand{\qed}{\textbf{\textit{QED}}}
\newcommand{\qef}{\textbf{\textit{QEF}}}
\newcommand{\task}[1]{\textit{#1}}
\newcommand{\xrowht}[2][0]{\addstackgap[.5\dimexpr#2\relax]{\vphantom{#1}}}

\DeclareMathOperator{\?}{?}
\DeclareMathOperator{\dom}{dom}
\DeclareMathOperator{\im}{im}
\DeclareMathOperator{\lin}{lin}
\DeclareMathOperator{\R}{\mathbb{R}}
\DeclareMathOperator{\Z}{\mathbb{Z}}

\let\oldepsilon\epsilon \renewcommand{\epsilon}{\varepsilon}

% Koniec dodatkowych deklaracji

\pagestyle{fancy}

\lhead{Analiza Matematyczna 2; 2020/2021}
% Tutaj proszę uzupełnić imię i nazwisko
\rhead{Jakub Łukasiewicz}

\begin{document}
\section*{Zadanie 3a}
\task{Korzystając z kryterium porównawczego zbadać zbieżność szeregu
$\displaystyle \sum_{n=1}^{\infty}{\frac{3}{n^2 + 2}}$}

\subsection*{Rozwiązanie}

Oznaczmy: $ \displaystyle a_n = \frac{3}{n^2 + 2} \ge 0 \for n \ge 1 $

\[ b_n = \frac{3}{n^2} = 3\frac{1}{n^2} \]
\[ b_n \ge a_b \for n \ge 1 \]
\[ \sum_{n=1}^{\infty}{b_n} = 3\sum_{n=1}^{\infty}{\frac{1}{n^2}} \]

Z \textit{faktu o zbieżności szeregów postaci $\sum_{n=1}^{\infty}{\frac{1}{n^p}}$}
wiemy, że taki szereg jest zbieżny dla $p > 1$. W ciągu $b_n$, wykładnik $p=2>1$, zatem
szereg ciągu $b_n$ jest zbieżny.

Stąd z \textit{kryterium porównawczego} wynika, że szereg
$\displaystyle \sum_{n=1}^{\infty}{\frac{3}{n^2 + 2}}$ jest również zbieżny.

\subsection*{Odpowiedź}
\centerline{Szereg $\displaystyle \sum_{n=1}^{\infty}{\frac{3}{n^2 + 2}}$ jest \textbf{zbieżny}}

\vspace{5em}

\section*{Zadanie 4d}
\task{Korzystając z kryterium d’Alemberta zbadać zbieżność szeregu
$\displaystyle \sum_{n=1}^{\infty}{\frac{n^n}{3^n\ n!}}$}

\subsection*{Rozwiązanie}

\begin{equation*}
   \begin{aligned}
         \lim_{n \to \infty} \left| \frac{a_{n+1}}{a_n} \right|
      &= \lim_{n \to \infty} \left| \frac{\frac{(n+1)^{n+1}}{3^{n+1}(n+1)!}}{\frac{n^n}{3^n\ n!}} \right| =
         \lim_{n \to \infty} \left| \frac{\frac{(n+1)^n (n+1)}{3\cdot 3^n(n+1)!}}{\frac{n^n}{3^n\ n!}} \right| =
         \lim_{n \to \infty} \left| \frac{(n+1)^n (n+1)}{3\cdot 3^n(n+1)!} \cdot \frac{3^n\ n!}{n^n} \right| =\\
      &= \lim_{n \to \infty} \left| \frac{(n+1)^n (n+1)}{3\cdot (n+1)} \cdot \frac{1}{n^n} \right| =
         \lim_{n \to \infty} \left| \frac{(n+1)^n}{3n^n} \right| =
         \lim_{n \to \infty} \frac{1}{3} \left| \left(\frac{n+1}{n}\right)^n \right| =\\
      &= \lim_{n \to \infty} \frac{1}{3} \left| \left(\frac{n}{n} + \frac{1}{n}\right)^n \right| =
         \frac{1}{3} \lim_{n \to \infty} \left| \left(1 + \frac{1}{n}\right)^n \right| =
         \frac{1}{3} \left| 1 \right| = \frac{1}{3}
   \end{aligned}
\end{equation*}

$\displaystyle \lim_{n \to \infty} \left| \frac{a_{n+1}}{a_n} \right | = \frac{1}{3} < 1 $,
zatem z \textit{kryterium d'Alemberta} wynika, że szereg jest zbieżny.

\subsection*{Odpowiedź}
\centerline{Szereg $\displaystyle \sum_{n=1}^{\infty}{\frac{n^n}{3^n\ n!}}$ jest \textbf{zbieżny}}

\clearpage

\section*{Zadanie 5c}
\task{Korzystając z kryterium Cauchy’ego zbadać zbieżność szeregu
$\displaystyle \sum_{n=1}^{\infty}{\arccos^n{\frac{1}{n^2}}} $}

\subsection*{Rozwiązanie}

\begin{equation*}
      \lim_{n \to \infty} \sqrt[n]{|a_n|} =
      \lim_{n \to \infty} \sqrt[n]{\left|\arccos^n{\frac{1}{n^2}}\right|} =
      \lim_{n \to \infty} \left|\arccos{\frac{1}{n^2}}\right| =
      |\arccos{0}| = \left|\frac{\pi}{2}\right| = \frac{\pi}{2} > 1
\end{equation*}
$\displaystyle \lim_{n \to \infty} \sqrt[n]{|a_n|} = \frac{\pi}{2} > 1 $, zatem z \textit{kryterium
Cauchy'ego} wynika, że szereg jest rozbieżny.

\subsection*{Odpowiedź}
\centerline{Szereg $\displaystyle \sum_{n=1}^{\infty}{\arccos^n{\frac{1}{n^2}}} $
jest \textbf{rozbieżny}}

\end{document}
