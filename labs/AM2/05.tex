\documentclass{article}
\usepackage[a4paper, margin={1in, 1in}]{geometry}
\usepackage[utf8]{inputenc}
\usepackage{polski}
\usepackage{mathtools}
\usepackage{amsfonts}
\usepackage{amssymb}
\usepackage{amsmath}
\usepackage{multicol}
\usepackage{paralist}
\usepackage{tabto}
\usepackage{graphicx}
\usepackage{etoolbox}
\usepackage{changepage}
\usepackage{tasks}
\usepackage{pgfplots}
\usepackage{fancyhdr}

\DeclareMathOperator{\arctg}{arctg}
\DeclareMathOperator{\sh}{sh}
\DeclareMathOperator{\ch}{ch}
\DeclareMathOperator{\sgn}{sgn}

\let\arctan\relax
\DeclareMathOperator{\arctan}{arctg}
\let\tan\relax
\DeclareMathOperator{\tan}{tg}

% Dodatkowe deklaracje:

\newcommand{\Because}{\quad\because\quad}
\newcommand{\case}[1]{\textrm{#1$^\circ$}}
\newcommand{\degree}{^{\circ}}
\newcommand{\diff}{\mathop{}\!\mathrm{d}}
\newcommand{\for}{\ \leftrightarrow\ }
\newcommand{\Integral}[4]{\int_{#1}^{#2} \! #3 \, \mathop{}\!\mathrm{d}#4}
\newcommand{\mathcolorbox}[2]{\colorbox{#1}{$\displaystyle #2$}}
\newcommand{\norm}[1]{\left\lVert#1\right\rVert}
\newcommand{\qed}{\textbf{\textit{QED}}}
\newcommand{\qef}{\textbf{\textit{QEF}}}
\newcommand{\xrowht}[2][0]{\addstackgap[.5\dimexpr#2\relax]{\vphantom{#1}}}

\DeclareMathOperator{\?}{?}
\DeclareMathOperator{\dom}{dom}
\DeclareMathOperator{\im}{im}
\DeclareMathOperator{\lin}{lin}
\DeclareMathOperator{\N}{\mathbb{N}}
\DeclareMathOperator{\R}{\mathbb{R}}
\DeclareMathOperator{\Z}{\mathbb{Z}}

\let\oldepsilon\epsilon \renewcommand{\epsilon}{\varepsilon}

% Koniec dodatkowych deklaracji

\pagestyle{fancy}

\lhead{Analiza Matematyczna 2; 2020/2021}
% Tutaj proszę uzupełnić imię i nazwisko
\rhead{Jakub Łukasiewicz}

\begin{document}
\section*{Zadanie 1d}
Obliczyć wszystkie pochodne cząstkowe pierwszego rzędu funkcji
$f(x,y,z) = \sin{\big(x \cos{(y \sin{z})}\big)}$.

\subsection*{Rozwiązanie}

Rozwiązanie opiera się na:
\begin{itemize}
   \item \textit{Przy obliczaniu pochodnej cząstkowej względem jednej zmiennej, pozostałe traktowane są jak stałe}
   \item $(g \circ f)'(x) = g'\Big(f(x)\Big) f'(x)$
\end{itemize}

\begin{equation*}
   \frac{\partial f(x,y,z)}{\partial x} = \Big| a = \cos{(y \sin{z})} \Big| =
   \frac{\partial}{\partial x}(\sin{ax}) = (\sin{ax})' = (ax)' \cdot \cos(ax) = a \cos(ax)
\end{equation*}

\begin{equation*}
   \frac{\partial f(x,y,z)}{\partial y} =
   \left| \begin{aligned}
      f(x,y,z) &= F(G(H(I(y)))) \\
      F(y)     &= \sin y \\
      G(y)     &= xy \\
      H(y)     &= \cos y \\
      I(y)     &= y \sin z
   \end{aligned} \right|
   = -x \sin{\big(y \sin z\big)} \cos{\big(x \cos{(y \sin z)} \big)} \sin{z}
\end{equation*}

\begin{equation*}
   \frac{\partial f(x,y,z)}{\partial z} =
   \left| \begin{aligned}
      f(x,y,z) &= F(G(H(I(J(z))))) \\
      F(z)     &= \sin z \\
      G(z)     &= xz \\
      H(z)     &= \cos z \\
      I(z)     &= yz \\
      J(z)     &= \sin z
   \end{aligned} \right|
   = -xy \sin{\big(y \sin z \big)} \cos{\big(x \cos{(y \sin z)} \big)} \cos{z}
\end{equation*}

\subsection*{Odpowiedź}
\begin{align*}
   \frac{\partial f(x,y,z)}{\partial y} = \cos{\big(y \sin{z}\big)} \cos\big(x \cos{(y \sin{z})}\big)
\end{align*}
\begin{equation*}
   \frac{\partial f(x,y,z)}{\partial y} = -x \sin{\big(y \sin z\big)} \cos{\big(x \cos{(y \sin z)} \big)} \sin{z}
\end{equation*}
\begin{equation*}
   \frac{\partial f(x,y,z)}{\partial z} = -xy \sin{\big(y \sin z \big)} \cos{\big(x \cos{(y \sin z)} \big)} \cos{z}
\end{equation*}

\vspace{3em}

\section*{Zadanie 2a}
Korzystając z definicji zbadać, czy istnieją pochodne cząstkowe rzędu pierwszego
funkcji \\ $\displaystyle f(x,y) = \left\{ \begin{array}{ll}
   x^2 + y^2 & \for xy = 0 \\
   1         & \for xy \ne 0
\end{array}\right. $ w punkcie $(x_0, y_0) = (0,0)$.

\subsection*{Rozwiązanie}

\begin{equation*}
   \frac{\partial f(0,0)}{\partial x} = \lim_{\Delta{x} \to 0} \frac{f(0+\Delta{x},0) - f(0,0)}{\Delta{x}} =
   \lim_{\Delta{x} \to 0} \frac{(\Delta{x})^2 - 0}{\Delta{x}} =
   \lim_{\Delta{x} \to 0} \Delta{x} = 0
\end{equation*}

\begin{equation*}
   \frac{\partial f(0,0)}{\partial y} = \lim_{\Delta{y} \to 0} \frac{f(0, 0+\Delta{y}) - f(0,0)}{\Delta{y}} =
   \lim_{\Delta{x} \to 0} \frac{(\Delta{y})^2 - 0}{\Delta{y}} =
   \lim_{\Delta{y} \to 0} \Delta{y} = 0
\end{equation*}

\vspace{3em}

\end{document}
