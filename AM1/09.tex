% LaTeX

\documentclass[12pt]{article}

\usepackage[a4paper, vmargin=10pt, hmargin=50pt]{geometry}
\usepackage{amsmath}
\usepackage{amssymb}
\usepackage{pgfplots}
\usepackage{polski}
\usepackage{xcolor}

\setlength{\parskip}{1em}

\newcommand{\degree}{^{\circ}}
\newcommand{\ex}[1]{\textbf{ #1)}}
\newcommand{\norm}[1]{\left\lVert#1\right\rVert}
\newcommand{\qed}{\textbf{\textit{QED}}}
\newcommand{\qef}{\textbf{\textit{QEF}}}
\newcommand{\task}[1]{\section*{Zadanie #1}}
\newcommand{\mathcolorbox}[2]{\colorbox{#1}{$\displaystyle #2$}}

\let\oldref\ref
\renewcommand{\ref}[1]{(\oldref{#1})}

\DeclareMathOperator{\dom}{dom}
\DeclareMathOperator{\sgn}{sgn}
\DeclareMathOperator{\R}{\mathbb{R}}

\renewcommand{\epsilon}{\varepsilon}

\title{Zestaw 9}
\author{Jakub Łukasiewicz}
\date{}

\begin{document}

\maketitle

\task{1}\ex{b}
Znaleźć asymptoty pionowe i ukośne funkcji $g(x) = \frac{x^3}{(x+1)^2}$.

$$ \dom{g} = \R \setminus \{-1\} $$
\begin{equation*}
    \lim_{x \to -1^-}{g} = \left[ \frac{1}{0^+} \right] = \infty
    \quad\land\quad
    \lim_{x \to -1^+}{g} = \left[ \frac{1}{0^+} \right] = \infty
    \implies \textrm{asymptota pionowa: } \mathcolorbox{yellow}{x = -1}
\end{equation*}
\begin{equation} \label{a}
    a = \lim_{x \to \pm\infty}{\frac{g(x)}{x}} = \lim_{x \to \pm\infty}{\frac{x^2}{(x+1)^2}}
    = \lim_{x \to \pm\infty}{\frac{x^2}{x^2 + 2x + 1}} = 1
\end{equation}
\begin{equation} \label{b}
    \ref{a} \implies b = \lim_{x \to \pm\infty}{g(x) - 1x}
    = \lim_{x \to \pm\infty}{\frac{x^3}{(x+1)^2} - x}
    = \lim_{x \to \pm\infty}{\frac{x^3 - x^3 - 2x^2 - x}{x^2 + 2x + 1}} = -2
\end{equation}
$$ \ref{a} \land \ref{b} \implies \textrm{asymptota ukośna: } \mathcolorbox{yellow}{y = x - 2} $$

\task{2}\ex{f}
Narysować wykres funkcji $r$ takiej, że: \textit{jest parzysta}, $\lim\limits_{x \to 0^-}{r(x)} = \infty$
i $\lim\limits_{x \to \infty}{[r(x) - 1]} = -1$.

Przykładem funkcji spełniającej podane założenia jest $ \left| \frac{10}{x} \right| $ o wykresie:

\begin{figure}[h!]
    \centering
    \begin{tikzpicture}[>=stealth]
        \begin{axis}[
                xmin=-100,xmax=100,
                ymin=-1,ymax=4,
                axis x line=middle,
                axis y line=middle,
                axis line style=->,
                xlabel={$x$},
                ylabel={$y$},
            ]
            \addplot[blue] expression[domain=-100:100,samples=300]{10/abs(x)};
        \end{axis}
    \end{tikzpicture}
\end{figure}

\newpage

\setcounter{equation}{0}
\task{3}\ex{e}
Zbadać przebieg zmienności funkcji $q(x) = x^2 e^{-x}$ i następnie sporządzić jej wykres.

$$ q(x) = x^2 e^{-x} = \frac{x^2}{e^x} $$
$$ \dom{q} = \R $$
$$ q(\dom{q}) = [0, \infty ) $$
$$ q(x_0) = 0 \leftrightarrow x_0 = 0 $$
$$ q(0) = 0 $$

$$ \lim_{x \to -\infty}{q(x)} = \lim_{x \to -\infty}{x^2e^x} = \infty $$
%
$$
\lim_{x \to \infty}{q(x)} =
\lim_{x \to \infty}{\frac{x^2}{e^x}} =
\lim_{x \to \infty}{\frac{(x^2)'}{(e^x)'}} =
\lim_{x \to \infty}{\frac{2x}{e^x}} =
\lim_{x \to \infty}{\frac{(2x)'}{(e^x)'}} =
\lim_{x \to \infty}{\frac{2}{e^x}} =
0
$$

$$ q'(x) = \left(\frac{x^2}{e^x}\right)' = \frac{2xe^x - x^2e^x}{e^{2x}}
= \frac{xe^x(2-x)}{e^xe^x} = \frac{x(2-x)}{e^x} $$

\begin{equation} \label{qe0}
    q'(x) = 0 \Leftrightarrow \frac{x(2-x)}{e^x} = 0 \Leftrightarrow x = 0 \lor x = 2
\end{equation}
\begin{equation} \label{qgt0}
    q'(x) > 0 \Leftrightarrow \frac{x(2-x)}{e^x} > 0 \Leftrightarrow
    x(2-x) > 0 \Leftrightarrow x \in (0,2)
\end{equation}
\begin{equation} \label{qlt0}
    q'(x) < 0 \Leftrightarrow x \in (-\infty,0)\cup(2,\infty)
\end{equation}

$$ \ref{qe0} \land \ref{qgt0} \land \ref{qlt0} \implies
\left\{ \begin{array}{ll}
    \textrm{minimum  lokalne: } 0\\
    \textrm{maksimum lokalne: } 2
\end{array} \right.
$$

$$ q''(x) = \frac{e^x(2x-x^2)' - (2x-x^2)e^x}{e^xe^x} =
\frac{2 - 2x - 2x + x^2}{e^x} = \frac{x^2 - 4x + 2}{e^x} $$

$$ q''(x) = 0 \Leftrightarrow x^2 - 4x + 2 = 0 \Leftrightarrow
x \in \left\{ 2 - \sqrt{2}, 2 + \sqrt{2} \right\} $$
%
$$ q''(x) > 0 \Leftrightarrow x \in (-\infty, 2-\sqrt{2})\cup(2+\sqrt{2}, \infty)$$
%
$$ q''(x) < 0 \Leftrightarrow x \in (2-\sqrt{2}, 2+\sqrt{2}) $$

\subsubsection*{Podsumowanie:}
\begin{itemize}
    \item[] Domena: $\R$
    \item[] Zbiór wartości: $[0, \infty)$
    \item[] Miejsca zerowe: $0$
    \item[] Punkt przecięcia z osią Oy: $0$
    \item[] Granice:
        \begin{itemize}
            \item[] lewostronna: $\infty$
            \item[] prawostronna: $0$
        \end{itemize}
    \item[] Asymptoty: pozioma $y=0$
    \item[] Monotoniczność:
        \begin{itemize}
            \item[] $q \searrow$ $(-\infty,0)$, $(2,\infty)$;
            \item[] $q \nearrow$ $(0,2)$
        \end{itemize}
    \item[] Ekstrema lokalne:
        \begin{itemize}
            \item[] minimum:  0
            \item[] maksimum: 2
        \end{itemize}
    \item[] Przedziały wypukłości:
        \begin{itemize}
            \item[] wklęsłość: $(2-\sqrt{2}, 2+\sqrt{2})$
            \item[] wypukłość: $(-\infty, 2-\sqrt{2})$, $(2+\sqrt{2}, \infty)$
        \end{itemize}
    \item[] Punkty przegięcia: $2 - \sqrt{2}$, $2 + \sqrt{2}$
    \item[] Wykres:
        \begin{figure}[h!]
            \centering
            \begin{tikzpicture}[>=stealth]
                \begin{axis}[
                        xmin=-2,xmax=5,
                        ymin=-0.5,ymax=5,
                        axis x line=middle,
                        axis y line=middle,
                        axis line style=->,
                        xlabel={$x$},
                        ylabel={$y$},
                    ]
                    \addplot[blue] expression[domain=-3:5,samples=300]{x*x/e^x};
                \end{axis}
            \end{tikzpicture}
        \end{figure}
\end{itemize}

\end{document}
