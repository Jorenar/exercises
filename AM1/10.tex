% LaTeX

\documentclass[12pt]{article}

\usepackage[a4paper, vmargin=10pt, hmargin=50pt]{geometry}
\usepackage{amsmath}
\usepackage{amssymb}
\usepackage{multicol}
\usepackage{polski}
\usepackage{xcolor}

\setlength{\parskip}{1em}

\newcommand{\degree}{^{\circ}}
\newcommand{\ex}[1]{\textbf{ #1)}}
\newcommand{\for}{\ \leftrightarrow\ }
\newcommand{\Integral}[2]{\int \! #1 \, \mathrm{d}#2}
\newcommand{\mathcolorbox}[2]{\colorbox{#1}{$\displaystyle #2$}}
\newcommand{\norm}[1]{\left\lVert#1\right\rVert}
\newcommand{\qed}{\textbf{\textit{QED}}}
\newcommand{\qef}{\textbf{\textit{QEF}}}
\newcommand{\task}[1]{\section*{Zadanie #1}}

\let\oldref\ref
\renewcommand{\ref}[1]{(\oldref{#1})}

\DeclareMathOperator{\dom}{dom}
\DeclareMathOperator{\sgn}{sgn}
\DeclareMathOperator{\R}{\mathbb{R}}

\renewcommand{\epsilon}{\varepsilon}

\title{Zestaw 10}
\author{Jakub Łukasiewicz}
\date{}

\begin{document}

\maketitle

\task{2}\ex{e}
Korzystając z twierdzenia o całkowaniu przez części obliczyć całkę nieoznaczoną
$ \displaystyle \Integral{\arccos^2{x}}{x} $

$$ (\arccos{x})' = -\frac{1}{\sqrt{1-x^2}} $$

\begin{multicols}{2}
    $$ f(x) = \arccos^2{x} $$
    $$ f'(x) = -\frac{2\arccos{x} }{\sqrt{1-x^2}} $$

    \columnbreak

    $$ g'(x) = 1 $$
    $$ g(x) = x $$
\end{multicols}

\begin{equation} \label{int1}
    \begin{aligned}
        \Integral{\arccos^2{x}}{x} =
        & = \Integral{f(x)g'(x)}{x} = f(x)g(x) - \Integral{f'(x)g(x)}{x} =\\
        & = (\arccos^2{x})(x) - \Integral{-\frac{2\arccos{x} }{\sqrt{1-x^2}}x}{x} =\\
        & = (\arccos^2{x})(x) - 2\Integral{-\frac{\arccos{x}}{\sqrt{1-x^2}}x}{x}
    \end{aligned}
\end{equation}

$$ t = \arccos{x} $$
\begin{equation} \label{int2}
    \begin{aligned}
        \Integral{-\frac{\arccos{x}}{\sqrt{1-x^2}}x}{x} &= -\Integral{-t \cos{t}}{t} =
        \Integral{t \cos{t}}{t} = \\
        & = t \sin{t} - \Integral{\sin{t}}{t} = t\sin{t} + \cos{t} = \\
        & = \arccos{x} \cdot \sin{(\arccos{x})} + \cos{(\arccos{x})} = \\
        & = (\arccos{x})\sqrt{1-x^2} + x
    \end{aligned}
\end{equation}

\begin{equation*}
    \ref{int1} \land \ref{int2} \implies
    \Integral{\arccos^2{x}}{x} =
    (\arccos^2{x})(x) - 2((\arccos{x})\sqrt{1-x^2} + x)
\end{equation*}

$$ \displaystyle \Integral{\arccos^2{x}}{x} = x \arccos^2{x} - 2(\arccos{x})\sqrt{1-x^2} - 2x + C \for C\in\R $$
$$\qef$$

\newpage

\setcounter{equation}{0}
\task{3}\ex{h}
Stosując odpowiednie podstawienia obliczyć całkę nieoznaczoną
$ \displaystyle \Integral{\frac{5 \sin{x}}{3 - 2 \cos{x}}}{x} $

\begin{equation} \label{int2_1}
    \Integral{\frac{5 \sin{x}}{3 - 2 \cos{x}}}{x} =
    5 \Integral{\frac{\sin{x}}{3 - 2 \cos{x}}}{x} =
    5 \Integral{\frac{t}{3-2t}}{t} \for
    \left\{ \begin{array}{l}
        t = \cos{x} \\
        \mathrm{d}t = \sin{x}
    \end{array} \right.
\end{equation}
\begin{equation} \label{int2_2}
    \Integral{\frac{t}{3-2t}}{t} = \Integral{-\frac{1}{2u}}{u} \for
    \left\{ \begin{array}{l}
        u = 3 - 2t\\
        \mathrm{d}u = -\frac{1}{2}{\mathrm{d}t}
    \end{array} \right.
\end{equation}
\begin{equation*}
    \begin{aligned}
        \ref{int2_1} \land \ref{int2_2} \implies
        \Integral{\frac{5 \sin{x}}{3 - 2 \cos{x}}}{x} &= 5\Integral{-\frac{1}{2u}}{u} =
         -\frac{5}{2} \Integral{\frac{1}{u}}{u} = \\
         & = -\frac{5}{2} \ln{u} =
         -\frac{5\ln{(3-2t)}}{2} =
         -\frac{5\ln{(3-2 \cos{x})}}{2}
    \end{aligned}
\end{equation*}

$$ \Integral{\frac{5 \sin{x}}{3 - 2 \cos{x}}}{x} = -\frac{5\ln{(3-2 \cos{x})}}{2} + C \for C \in \R $$
$$ \qef $$

\setcounter{equation}{0}
\task{4}\ex{a}
Obliczyć całkę nieoznaczoną $ \displaystyle \Integral{\left| 1-x^2 \right|}{x} $

\begin{equation} \label{int3_1}
    \Integral{\left| 1-x^2 \right|}{x} =
    \left\{ \begin{array}{l}
        \displaystyle \Integral{(1 - x^2)}{x} \for x \in [-1,1] \\
        \displaystyle \Integral{(x^2 - 1)}{x} \for x \in (-\infty,-1)\cap(1,\infty)
    \end{array} \right.
\end{equation}

\begin{equation} \label{int3_2}
    \Integral{(1 - x^2)}{x} = x - \frac{x^3}{3}
\end{equation}
\begin{equation} \label{int3_3}
    \Integral{(x^2 - 1)}{x} = \frac{x^3}{3} - x
\end{equation}

$$
\ref{int3_1} \land \ref{int3_2} \land \ref{int3_3} \implies
\Integral{\left| 1-x^2 \right|}{x} =
    \left\{ \begin{array}{l}
        \displaystyle x - \frac{x^3}{3} \for x \in [-1,1] \\
        \displaystyle \frac{x^3}{3} - x \for x \in (-\infty,-1)\cap(1,\infty)
    \end{array} \right.
$$

$$
\Integral{\left| 1-x^2 \right|}{x} =
    \left\{ \begin{array}{l}
        \displaystyle x - \frac{x^3}{3} + C \for x \in [-1,1] \\
        \displaystyle \frac{x^3}{3} - x + C \for x \in (-\infty,-1)\cap(1,\infty)
    \end{array} \right. \for C \in \R
$$
$$ \qef $$

\end{document}
