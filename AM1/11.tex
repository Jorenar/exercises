% LaTeX

\documentclass[12pt]{article}

\usepackage[a4paper, vmargin=10pt, hmargin=50pt]{geometry}
\usepackage{amsmath}
\usepackage{amssymb}
\usepackage{multicol}
\usepackage{polski}
\usepackage{xcolor}

\setlength{\parskip}{1em}

\newcommand{\degree}{^{\circ}}
\newcommand{\ex}[1]{\textbf{ #1)}}
\newcommand{\for}{\ \leftrightarrow\ }
\newcommand{\Integral}[2]{\int \! #1 \, \mathrm{d}#2}
\newcommand{\mathcolorbox}[2]{\colorbox{#1}{$\displaystyle #2$}}
\newcommand{\norm}[1]{\left\lVert#1\right\rVert}
\newcommand{\qed}{\textbf{\textit{QED}}}
\newcommand{\qef}{\textbf{\textit{QEF}}}
\newcommand{\task}[1]{\section*{Zadanie #1}}

\let\oldref\ref
\renewcommand{\ref}[1]{(\oldref{#1})}

\DeclareMathOperator{\dom}{dom}
\DeclareMathOperator{\sgn}{sgn}
\DeclareMathOperator{\R}{\mathbb{R}}

\renewcommand{\epsilon}{\varepsilon}

\title{Zestaw 11}
\author{Jakub Łukasiewicz}
\date{}

\begin{document}

\maketitle

\task{1}\ex{g}
\textbf{Obliczyć całkę z funkcji wymiernej: }
$ \displaystyle \Integral{\frac{x}{(x-1)(x+2)(x+3)}}{x} $

$$ \frac{x}{(x-1)(x+2)(x+3)} = \frac{A}{x-1} + \frac{B}{x+2} + \frac{C}{x+3} $$

\begin{equation*}
    \begin{aligned}
        \frac{A}{x-1} + \frac{B}{x+2} + \frac{C}{x+3} &=\\
        &= \frac{A(x+2)(x+3) + B(x-1)(x+3) + C(x-1)(x+2)}{(x-1)(x+2)(x+3)} =\\
        &= \frac{(A+B+C)x^2 + (5A+2B+C)x + (6A-3B-2C)}{(x-1)(x+2)(x+3)}
    \end{aligned}
\end{equation*}

\begin{equation*}
    \frac{x}{(x-1)(x+2)(x+3)} = \frac{(A+B+C)x^2 + (5A+2B+C)x + (6A-3B-2C)}{(x-1)(x+2)(x+3)} \implies
\end{equation*}

\begin{equation*}
    \implies \left\{ \begin{array}{l}
        A+B+C = 0        \\
        5A + 2B + C = 1  \\
        6A - 3B - 2C = 0
    \end{array} \right. \implies
    \def\arraystretch{1.2}
    \left\{ \begin{array}{l}
        A = \frac{1}{12} \\
        B = \frac{2}{3}  \\
        C = -\frac{3}{4}
    \end{array} \right. \implies
\end{equation*}

\begin{equation*}
    \implies \Integral{\frac{x}{(x-1)(x+2)(x+3)}}{x} =
    \Integral{\left(\frac{\frac{1}{12}}{x-1} + \frac{\frac{2}{3}}{x+2} + \frac{-\frac{3}{4}}{x+3}\right)}{x} =
\end{equation*}

\begin{equation*}
    = \frac{1}{12}\Integral{\frac{1}{x-1}}{x}
    + \frac{2}{3}\Integral{\frac{1}{x+2}}{x}
    - \frac{3}{4}\Integral{\frac{1}{x+3}}{x}
    \implies
\end{equation*}

\begin{equation*}
    \implies \Integral{\frac{x}{(x-1)(x+2)(x+3)}}{x} =
    \frac{\ln{|x-1|} + 8 \ln{|x+2|} - 9 \ln{|x+3|}}{12}
\end{equation*}

$$\qef$$


\newpage

\setcounter{equation}{0}
\task{2}\ex{d}
\textbf{Obliczyć całkę z funkcji trygonometrycznej: }
$ \displaystyle \Integral{\cos^4{x}}{x} $

$$ \cos{2x} = 2 \cos^2{x} - 1 \implies cos^2{x} = \frac{1 + \cos{2x}}{2} $$

\begin{equation*}
    \begin{aligned}
        \Integral{\cos^4{x}}{x} &
        = \Integral{(\cos^2{x})^2}{x}
        = \Integral{\left(\frac{1 + \cos{2x}}{2}\right)^2}{x}
        = \Integral{\frac{(1 + \cos{2x})^2}{4}}{x} =\\
        &= \frac{1}{4} \Integral{(1 + 2\cos{2x} + \cos^2{2x})}{x}
        = \frac{1}{4} \left( \Integral{1}{x} + 2\Integral{\cos{2x}}{x} + \Integral{\cos^2{2x}}{x} \right) =\\
        &= \frac{x}{4} + \frac{\frac{1}{2} \sin{2x}}{2} + \frac{1}{4}\Integral{\cos^2{2x}}{x}
        = \frac{x + \sin{2}}{4} + \frac{1}{4}\Integral{\cos^2{2x}}{x} =\\
        &= \frac{x + \sin{2}}{4} + \frac{1}{4}\Integral{\frac{1 + \cos{4x}}{2}}{x} =
        = \frac{x + \sin{2x}}{4} + \frac{x}{8} + \frac{1}{8}\Integral{\cos{4x}}{x} =\\
        &= \frac{3x + 2\sin{2x}}{8} + \frac{1}{8}\Integral{\cos{4x}}{x} =
        = \frac{3x + 2\sin{2x}}{8} + \frac{1}{8} \cdot \frac{1}{4} \sin{4x} =\\
        &= \frac{12x + 8\sin{2x} + \sin{4x}}{32}
        \\ \\
        \Integral{\cos^4{x}}{x} &= \frac{12x + 8\sin{2x} + \sin{4x}}{32} + C \for C \in \R \qquad \qef
    \end{aligned}
\end{equation*}


\setcounter{equation}{0}
\task{3}\ex{d}
\textbf{Obliczyć całkę z funkcji niewymiernej: }
$ \displaystyle \Integral{\sqrt{3+x^2}}{x} $

\begin{equation*}
    \begin{aligned}
        \Integral{\sqrt{3+x^2}}{x} &= \Integral{\sqrt{(\pm\sqrt{3})^2+x^2}}{x} =\\
        &= \frac{x}{2}\sqrt{x^2 + (\pm\sqrt{3})^2} + \frac{(\pm\sqrt{3})^2}{2} \ln{\left| x + \sqrt{x^2 + (\pm\sqrt{3})^2}\right|} =\\
        &= \frac{x}{2}\sqrt{x^2 + 3} + \frac{3}{2} \ln{\left| x + \sqrt{x^2 + 3}\right|}
    \end{aligned}
\end{equation*}

\begin{equation*}
        \Integral{\sqrt{3+x^2}}{x} = \frac{x}{2}\sqrt{x^2 + 3} + \frac{3}{2} \ln{\left| x + \sqrt{x^2 + 3}\right|} + C \for C \in \R
\end{equation*}

$$ \qef $$

\end{document}
