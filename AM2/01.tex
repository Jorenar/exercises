\documentclass{article}
\usepackage[a4paper, margin={1in, 1in}]{geometry}
\usepackage[utf8]{inputenc}
\usepackage{polski}
\usepackage{mathtools}
\usepackage{amsfonts}
\usepackage{amssymb}
\usepackage{amsmath}
\usepackage{multicol}
\usepackage{paralist}
\usepackage{tabto}
\usepackage{graphicx}
\usepackage{etoolbox}
\usepackage{changepage}
\usepackage{tasks}
\usepackage{pgfplots}
\usepackage{fancyhdr}

\DeclareMathOperator{\arctg}{arctg}
\DeclareMathOperator{\sh}{sh}
\DeclareMathOperator{\ch}{ch}
\DeclareMathOperator{\sgn}{sgn}

\let\arctan\relax
\DeclareMathOperator{\arctan}{arctg}
\let\tan\relax
\DeclareMathOperator{\tan}{tg}

% Dodatkowe deklaracje:

\newcommand{\case}[1]{\textrm{#1$^\circ$}}
\newcommand{\degree}{^{\circ}}
\newcommand{\diff}{\mathop{}\!\mathrm{d}}
\newcommand{\for}{\ \leftrightarrow\ }
\newcommand{\Integral}[4]{\int_{#1}^{#2} \! #3 \, \mathop{}\!\mathrm{d}#4}
\newcommand{\mathcolorbox}[2]{\colorbox{#1}{$\displaystyle #2$}}
\newcommand{\norm}[1]{\left\lVert#1\right\rVert}
\newcommand{\qed}{\textbf{\textit{QED}}}
\newcommand{\qef}{\textbf{\textit{QEF}}}
\newcommand{\task}[1]{\textbf{#1}}
\newcommand{\xrowht}[2][0]{\addstackgap[.5\dimexpr#2\relax]{\vphantom{#1}}}

\let\oldref\ref \renewcommand{\ref}[1]{\mathrm{(\oldref{#1})}}

\DeclareMathOperator{\?}{?}
\DeclareMathOperator{\dom}{dom}
\DeclareMathOperator{\im}{im}
\DeclareMathOperator{\lin}{lin}
\DeclareMathOperator{\R}{\mathbb{R}}
\DeclareMathOperator{\Z}{\mathbb{Z}}

\let\oldepsilon\epsilon \renewcommand{\epsilon}{\varepsilon}

\counterwithin*{equation}{section}  % reset equation counter after each section

\pagestyle{fancy}

\lhead{Analiza Matematyczna 2; 2020/2021}
% Tutaj proszę uzupełnić imię i nazwisko
\rhead{Jakub Łukasiewicz}

\begin{document}
\section*{Zadanie 1b}

\textit{Korzystając z definicji zbadać zbieżność całki niewłaściwej pierwszego rodzaju $\displaystyle \Integral{0}{\infty}{2^{-x}}{x}$}

\subsection*{Rozwiązanie}

\begin{equation} \label{1b1}
   \Integral{0}{\infty}{2^{-x}}{x} = \lim_{T \to \infty}{\Integral{0}{T}{2^{-x}}{x}}
\end{equation}

\begin{equation} \label{1b2}
   \Integral{}{}{2^{-x}}{x} =
   \begin{vmatrix}
      t = -x \\
      \diff{t} = -1 \diff{x} \\
      \diff{x} = -\diff{t}
   \end{vmatrix}
   = -\Integral{}{}{2^{t}}{t}
   = -\frac{2^{t}}{\ln{2}}
   = -\frac{2^{-x}}{\ln{2}} + C
\end{equation}

\begin{equation*}
   \begin{aligned}
      \ref{1b1} \land \ref{1b2} \implies \Integral{0}{\infty}{2^{-x}}{x} & = \lim_{T \to \infty} \left[ -\frac{2^{-x}}{\ln{2}} \right]_0^{T} =
      \lim_{T \to \infty}{ \left( -\frac{2^{-T}}{\ln{2}} + \frac{2^{-0}}{\ln{2}} \right) } =
      \lim_{T \to \infty}{ \left( \frac{-2^{-T} + 1}{\ln{2}} \right) } = \\
      &= \frac{1}{\ln{2}} \ \lim_{T \to \infty}{ \left( \frac{-1}{2^T} + 1 \right) } =
      \frac{0 + 1}{\ln{2}} = \frac{1}{\ln{2}}
   \end{aligned}
\end{equation*}

\subsection*{Odpowiedź}

\centerline{Całka $\displaystyle \Integral{0}{\infty}{2^{-x}}{x}$ jest
\textbf{zbieżna} do $\displaystyle \frac{1}{\ln{2}}$}

\clearpage

\section*{Zadanie 3b}

\textit{Zbadać zbieżność i zbieżność bezwzględną całki niewłaściwej $\displaystyle \Integral{\pi}{\infty}{x \cos{2x}}{x}$}

\subsection*{Rozwiązanie}

\setcounter{equation}{0}

\begin{equation} \label{1}
   \Integral{\pi}{\infty}{x \cos{2x}}{x} = \lim_{T \to \infty} \Integral{\pi}{T}{x \cos{2x}}{x}
\end{equation}

\begin{equation} \label{2}
   \begin{aligned}
      \Integral{}{}{x \cos{2x}}{x} &=
      \Integral{}{}{\frac{2x}{2} \cos{2x}}{x} =
      \begin{vmatrix}
         t = 2x \\[6pt]
         \diff{t} = 2 \diff{x} \\[6pt]
         \displaystyle \diff{x} = \frac{\diff{t}}{2}
      \end{vmatrix} =
      \Integral{}{}{\frac{t}{4} \cos{t}}{t} = \frac{1}{4}\Integral{}{}{t \cos{t}}{t} = \\
      &= \frac{1}{4}\Integral{}{}{t (\sin{t})'}{t} =
      \frac{1}{4}\left( t\sin{t} - \Integral{}{}{\sin{t}}{t} \right) =
      \frac{1}{4}\left( t\sin{t} - (-\cos{t}) \right) = \\
      &= \frac{1}{4}\left( t\sin{t} - (-\cos{t}) \right) = \frac{1}{4}\left( t\sin{t} + \cos{t} \right) =
      \frac{1}{4}\left( t\sin{t} + \cos{t} \right) =\\
      &= \frac{1}{4}\left( 2x\sin{2x} + \cos{2x} \right) + C
   \end{aligned}
\end{equation}

\begin{equation}
   \ref{1} \land \ref{2} \implies \Integral{\pi}{\infty}{x \cos{2x}}{x} =
   \lim_{T \to \infty} \left[ \frac{1}{4}\left( 2x\sin{2x} + \cos{2x} \right) \right]_\pi^{T}
   \in \varnothing
\end{equation}

\begin{figure}[h!]
   \centering
   \begin{tikzpicture}[>=stealth]
      \begin{axis}[
            xmin=-10,xmax=10,
            ymin=-30,ymax=30,
            axis x line=middle,
            axis y line=middle,
            axis line style=->,
            xlabel={$x$},
            ylabel={$y$},legend pos=outer north east
         ]
         \legend{$2x\sin{2x} + \cos{2x}$}
         \addplot[blue] expression[domain=-100:100,samples=1000]{2*x * sin(deg(2*x)) + cos(deg(2*x))};
      \end{axis}
   \end{tikzpicture}
\end{figure}

\subsection*{Odpowiedź}

Granica nie istnieje, zatem całka niewłaściwa jest \textbf{rozbieżna},
a z \textit{twierdzenia o zbieżności całek niewłaściwych zbieżnych bezwględnie} wynika, że
\textbf{nie} jest zbieżna bezwględnie.

\clearpage

\section*{Zadanie 5b}

\textit{Korzystając z kryterium porównawczego lub ilorazowego zbadać zbieżność całki
niewłaściwej drugiego rodzaju $\displaystyle \Integral{\pi}{2}{\frac{e^x}{x^3}}{x}$}

\textit{\tiny Źle przepisany przykład - powinno być: $\displaystyle \Integral{0}{2}{\frac{e^x}{x^3}}{x}$}

\subsection*{Rozwiązanie}

\setcounter{equation}{0}

\begin{equation} \label{5b1}
   \textrm{ Przyjmując w \textit{kryterium ilorazowym}:\quad}
   f(x) = \frac{1}{x^3} \quad\land\quad g(x) = \frac{e^x}{x^3}
\end{equation}

\begin{equation}
   \ref{5b1} \implies k = \lim_{x \to \pi^+}{\left( \frac{1}{x^3} \div \frac{e^x}{x^3} \right)}
     = \lim_{x \to \pi^+}{\left( \frac{1}{x^3} \cdot \frac{x^3}{e^x} \right)}
     = \lim_{x \to \pi^+}{\frac{1}{e^x}}
     = \frac{1}{e^\pi}
\end{equation}

\vspace{1.5em}

Sprawdzenie zbieżności całki $\Integral{\pi}{2}{f(x)}{x}$

\begin{equation}
   \Integral{\pi}{2}{\frac{1}{x^3}}{x} =
   \lim_{T \to 2}{\Integral{\pi}{T}{\frac{1}{x^3}}{x}} =
   \lim_{T \to 2}{\left[ -\frac{1}{2x^2} \right]_\pi^T} =
   \lim_{T \to 2}{\left( -\frac{1}{2T^2} + \frac{1}{2\pi^2} \right)} =
   -\frac{1}{8} + \frac{1}{2\pi^2} = \frac{4-\pi^2}{8\pi^2}
\end{equation}

\subsection*{Odpowiedź}

Całka $\displaystyle \Integral{\pi}{2}{\frac{1}{x^3}}{x} $ jest zbieżna oraz
$0 < k < \infty $, zatem badana całka jest również \textbf{zbieżna}

\end{document}
