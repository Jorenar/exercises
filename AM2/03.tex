\documentclass{article}
\usepackage[a4paper, margin={1in, 1in}]{geometry}
\usepackage[utf8]{inputenc}
\usepackage{polski}
\usepackage{mathtools}
\usepackage{amsfonts}
\usepackage{amssymb}
\usepackage{amsmath}
\usepackage{multicol}
\usepackage{paralist}
\usepackage{tabto}
\usepackage{graphicx}
\usepackage{etoolbox}
\usepackage{changepage}
\usepackage{tasks}
\usepackage{pgfplots}
\usepackage{fancyhdr}

\DeclareMathOperator{\arctg}{arctg}
\DeclareMathOperator{\sh}{sh}
\DeclareMathOperator{\ch}{ch}
\DeclareMathOperator{\sgn}{sgn}

\let\arctan\relax
\DeclareMathOperator{\arctan}{arctg}
\let\tan\relax
\DeclareMathOperator{\tan}{tg}

% Dodatkowe deklaracje:

\newcommand{\Because}{\quad\because\quad}
\newcommand{\case}[1]{\textrm{#1$^\circ$}}
\newcommand{\degree}{^{\circ}}
\newcommand{\diff}{\mathop{}\!\mathrm{d}}
\newcommand{\for}{\ \leftrightarrow\ }
\newcommand{\Integral}[4]{\int_{#1}^{#2} \! #3 \, \mathop{}\!\mathrm{d}#4}
\newcommand{\mathcolorbox}[2]{\colorbox{#1}{$\displaystyle #2$}}
\newcommand{\norm}[1]{\left\lVert#1\right\rVert}
\newcommand{\qed}{\textbf{\textit{QED}}}
\newcommand{\qef}{\textbf{\textit{QEF}}}
\newcommand{\task}[1]{\textit{#1}}
\newcommand{\xrowht}[2][0]{\addstackgap[.5\dimexpr#2\relax]{\vphantom{#1}}}

\DeclareMathOperator{\?}{?}
\DeclareMathOperator{\dom}{dom}
\DeclareMathOperator{\im}{im}
\DeclareMathOperator{\lin}{lin}
\DeclareMathOperator{\N}{\mathbb{N}}
\DeclareMathOperator{\R}{\mathbb{R}}
\DeclareMathOperator{\Z}{\mathbb{Z}}

\let\oldepsilon\epsilon \renewcommand{\epsilon}{\varepsilon}

\setlength{\parindent}{0em}

% Koniec dodatkowych deklaracji

\pagestyle{fancy}

\lhead{Analiza Matematyczna 2; 2020/2021}
% Tutaj proszę uzupełnić imię i nazwisko
\rhead{Jakub Łukasiewicz}

\begin{document}
\section*{Zadanie 4b}
\task{Znaleźć szereg Maclaurina funkcji $\displaystyle \cos{\frac{x}{2}}$
i określić przedziały jego zbieżności}

\subsection*{Rozwiązanie}

Korzystając ze wzoru $\displaystyle \cos{x} = \sum_{n=0}^{\infty} \frac{(-1)^n}{(2n)!} x^{2n} $ :

\vspace{1em}

\begin{equation*}
   \begin{aligned}
      \cos{\frac{x}{2}} &=
      \sum_{n=0}^{\infty} \frac{(-1)^n}{(2n)!} \left( \frac{x}{2} \right)^{2n} =
      \sum_{n=0}^{\infty} \frac{(-1)^n}{(2n)!} \frac{1}{2^{2n}} x^{2n} = \\
      &= \sum_{n=0}^{\infty} \frac{(-1)^n}{(2n)!} \frac{1}{4^n} x^{2n} =
      \sum_{n=0}^{\infty} \frac{(-\frac{1}{4})^n}{(2n)!} x^{2n} = \\
      &= 1 - \frac{x^2}{4 \cdot 2!} + \frac{x^4}{4^2 \cdot 4!} - \frac{x^6}{4^3 \cdot 6!} + ...
   \end{aligned}
\end{equation*}

gdzie $x \in \R$

\subsection*{Odpowiedź}
\[ \cos{\frac{x}{2}} = \sum_{n=0}^{\infty} \frac{(-\frac{1}{4})^n}{(2n)!} x^{2n} \ , \quad x \in \R \]

\clearpage

\section*{Zadanie 5d}
\task{Stosując tw. o różniczkowaniu i/lub całkowaniu szeregów potęgowych
obliczyć sumę szeregu $\displaystyle \sum_{n=1}^{\infty} \frac{n}{(n+2)2^n} $}

\subsection*{Rozwiązanie}

Ze wzoru na sumę nieskończonego ciągu geometrycznego wynika, że:
\begin{equation*}
   \sum_{n=1}^{\infty} t^{n+1} = \frac{t^2}{1-t} \for |t| < t
\end{equation*}

Całkując powyższe obustronnie na przedziale $[0,x]$ (gdzie $|x|<1$) wychodzi:
\begin{equation*}
   \sum_{n=1}^{\infty} \frac{x^{n+2}}{n+2} = -\frac{1}{2}x^2 - x - \ln(1-x)
\end{equation*}

Dzieląc obustronnie przez $x^2$ :
\begin{equation*}
   \sum_{n=1}^{\infty} \frac{x^n}{n+2} = -\frac{1}{2} - \frac{x + \ln(1-x)}{x^2} \mathrm{\ , \quad gdzie } |x| < 1
\end{equation*}

Dla $x=0$ po prawej stronie należy zastosować przejście graniczne $x \to 0$.
Natomiast różniczkując obustronnie powyższą tożsamość wychodzi:
\begin{equation*}
   \sum_{n=1}^{\infty} \frac{n}{n+2}x^{n-1} = \frac{x^2 + 2(1-x)(x + \ln(1-x))}{(1-x)x^3}
\end{equation*}

Zatem po pomnożeniu przez $x$ :
\begin{equation*}
   \sum_{n=1}^{\infty} \frac{n}{n+2}x^n = \frac{x^2 + 2(1-x)(x + \ln(1-x))}{(1-x)x^2}
\end{equation*}

Jeżeli za $x$ podstawi się $\frac{1}{2}$, to wychodzi wynik:
\begin{equation*}
   \sum_{n=1}^{\infty} \frac{n}{(n+2)2^n} = \frac{\frac{1}{4} + 1(\frac{1}{2} + \ln{\frac{1}{2}})}{\frac{1}{8}}
   = 6 + 8\ln{\frac{1}{2}}
\end{equation*}

\subsection*{Odpowiedź}
\[ \sum_{n=1}^{\infty} \frac{n}{(n+2)2^n} = 6 + 8\ln{\frac{1}{2}} \]

\end{document}
